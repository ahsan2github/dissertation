
\chapter{Conclusions and future directions}\label{chap:conclusion}
This dissertation adds new contributions to the existing scientific literature in regard to understanding Large and Very Large Scale Motions (LSMs \& VLSMs) in the atmospheric boundary layer. Along with the introduction of a new way to detect VLSMs and LSMs in simulated flow fields, more established methods and procedures have been utilized to study characteristics of these structures. In this chapter, we summarize the main findings and delineate the future research directions that can address questions that were not within the scope of this dissertation but demand new efforts to advance knowledge on VLSMs, and in understanding of the turbulent flow dynamics in general. 

\section{Summary}
While the existence of LSMs and VLSMs was predicted long before a true interest was observed by the scientific community in characterizing them, a lack of a universally accepted definition of VLSMs still plagues the research efforts to properly document their importance and characteristics. Visual cues have always been at the forefront of endeavours in regard to their identification in a flow field. A key contribution of this dissertation is the utilization of the visual characteristics of VLSMs and applying established tools and methods commonly used by the image processing community to identify these structures. In Chapter \ref{chap:chap1}, we developed a new definition of VLSMs as connected regions of pixels that correspond to a section of the probability density function of the velocity distribution and proceeded with the detection of such areas in the velocity field $u(\mathbf{x}, t)$ at a fixed $t$. Such a detection procedure enabled us to count the occurrences of VLSMs at any wall parallel plane and assess their contributions to shear stresses that can exclusively be attributed to VLSMs. It was observed that VLSMs do not necessarily make a disproportionately bigger contribution towards the shear stress compared to smaller scale counterparts. In fact, based on the area occupied by the VLSMs, the shear stress contribution of VLSMs was observed to be smaller in the log layer but higher in the outer layer of the BL in comparison to smaller scale structures. The projected 3D structure of VLSMs obtained using conditional averaging showed a difference between structure lengths and widths in different flow fields. It was observed that when rotation is significant in the flow fields, the large scale structures tend to shorten in length and widen in the cross-stream direction compared to those found in canonical pressure-driven channel flows. 

It has been previously reported from experimental investigations that VLSMs modulate the small-scale fluctuations. In order to examine whether these experimentally observed phenomena exist in the simulated flow fields and to solidify the fact that these phenomena are universal irrespective of boundary or forcing conditions, the modulation effect of VLSMs were analyzed in Chapter \ref{chap:chap2}. Although previous experiments investigated the modulation effect within the log layer only, strong evidence was found that the modulation effects extended beyond the log layer. In addition to the validation of the modulating effect, interscale energy exchange was analyzed to assess the locality and nonlocality of turbulent kinetic energy transfer for the three flow cases investigated. Any possibility of VLSMs interacting with small-scale motions in a nonlocal fashion due to nonlinearity of the Navier-Stokes equations would imply deviation from the Richardson-Kolmogorov proposed energy cascade scenario and also, that would mean VLSMs could directly modify small scale characteristics. The study of the locality and nonlocality of energy transfer for the very large scale motions poses some challenges due to the fact that the statistical tools available for analysis of turbulence may not be applied to study large-scale motions. As an example, the structure function-based formulations are not suitable for studying VLSMs because length-scale wise the applicability of structure function is limited to the inertial subrange. The $tke$ exchange between scales was analyzed in the wavelet domain. It was observed that energy transfer is predominantly local and weak rotation of the reference frame does not have an impact on $tke$ transfer characteristics. It can be concluded that in a domain where rotational effects are weak, small scale motions do not receive energy from distant large scales through inertial effects. Also the energy density of large scale in terms of $tke$ per unit length-scale per unit area were studied. While the total energy of VLSMs was found to be significant, the energy density per unit length-scale per unit area was found to be negligible compared to that of small-scale motions. 

In Chapter \ref{chap:chap3}, The time period of evolution of VLSMs was studied via Dynamic Mode Decomposition (DMD). Since the u-velocity component was the dominant contributor to the $tke$ and defined the overall structure in the velocity field,  an analysis of structures in the u-velocity component was undertakent. DMD ranks resultant modes based on frequency and a DMD mode corresponding to a particular frequency typically corresponds to a physical structure or pattern in the velocity field. An ensemble of 8000 successive frames were collected from the simulation and were analyzed resulting in 8000 modes per simulation. Because of our focus on VLSMs, modes with very slow frequencies were studied. Since VLSMs or LSMs correspond to outer scales, the frequency was converted to time period by normalizing with convection velocity at the top of the boundary and the boundary layer height. It was observed that structures of different length scales fell into the same frequency/time period bin. In an ideal case, only one dominant physical structure or pattern along with random noise is expected to correspond to a single time period. However, when analyzed, the high-Re flows showed that several patterns of different length scales correspond to a single mode. In physical context, this means that structures can be advected by larger scale motions without being dissipated by turbulence. The DMD method highlighted that the expected normalized time periods were longer than the expected time period calculated using the frozen turbulence hypothesis for VLSMs and LSMs. This indicates that the advection velocity of VLSMs and LSMs is slower than the mean flow. 

Actual flow in the environmen is subjected to a complex array of forcing such as Earth's rotation, surface heat flux, gravitational force, and pressure gradients and a multitude of surface conditions such as smooth sea surfaces, urban canopies to mountains, and everything in between. To study the effect of only one forcing condition on the flow, it is necessary to decouple the effect of the desired force or boundary condition from the ensemble. In such a case, simulating the flow becomes invaluable. Accounting for all these effects on VLSMs or LSMs would be a huge undertaking. Here, only the effect of rotation was considered. Evaluating the influence of other forcing and boundary conditions on VLSMs and LSMs is a necessary extension of this study. 


Many studies have focused on the importance and defining characteristics of VLSMs and LSMs. This present study also targets the identification and characterization of VLSMs and LSMs. One issue that must also be addressed is the dynamical development process of the large-scale motions. Several hypotheses have been proposed in that context. The most plausible hypothesis is that LSMs merge together to form VLSMs. LSMs come into existence due to counter rotating vortex rolls. These vortical structures usually extend towards the top of the boundary layer from the ground in the form of hairpin vortices. Such is the view of the attached eddy hypothesis. However, to solidify these arguments, an association must be established between the length scales characterizing the velocity field induced by a vortical structure and the characteristic diameter of the vortical structure. A  limited number of studies has focused in this direction. 
