%%% -*-LaTeX-*-
%%% This is the abstract for the thesis.
%%% It is included in the top-level LaTeX file with
%%%
%%%    \preface    {abstract} {Abstract}
%%%
%%% The first argument is the basename of this file, and the
%%% second is the title for this page, which is thus not
%%% included here.
%%%
%%% The text of this file should be about 350 words or less.

Turbulence in the Atmospheric Boundary Layer (ABL) is composed of a wide range of length
and time scales. To fully understand the turbulent dynamics of these motions in the ABL, it is
necessary to understand the interplay between these length and time scales and their
dependence on and interaction with different forcing and boundary conditions. Various studies
have confirmed the existence of Very Large Scale Motions (termed as “VLSMs”) in internal and
external flows and statistical properties of these large-scale motions have been cataloged.
However, how these structures or motions are affected throughout the ABL by realistic forcing
conditions where rotation plays a significant role has yet to be explored. Also, not well
understood is the interaction of VLSMs with smaller scales in regard to the turbulent kinetic
energy exchange. Aside from the dynamical significance of the VLSMs, the detection and
characterization of these structures is often not straightforward. In this, study a new detection
methodology was developed and used for the characterization of VLSMs in the ABL and
additionally, the turbulent kinetic energy exchange between large-scale and smaller scale
motions was studied quantitatively. The time scale of the VLSMs along with the challenge
associated with identifying the correct length scale is highlighted. It was found that any rotation
in the domain makes it difficult to identify the length scales of large-scale motions from velocity
component energy spectra. Rotation was also found to inhibit the spatial extent of VLSMs in the
primary wind direction while expanding it in the cross wind direction. However, given this, it is
somewhat surprising that rotation does not have a significant influence on the energy exchange
dynamics between scales. Finally, the spatial development of the large-scale motions and
related hypotheses have been revisited in the light of the obtained results.
