\chapter{Introduction}\label{chap:intro_chap}
The lower ten percent of the entire atmosphere is commonly known as the Atmospheric Boundary Layer (ABL). Most living organisms are engulfed by the ABL and are affected by the mixing processes in the ABL that dictate the distribution of particles e.g. moisture, carbon, and pollutants. The large-scale mixing processes in the ABL are mostly governed by turbulence. Thus, understanding the turbulence dynamics in the ABL has always been pivotal to endeavours oriented towards improving living conditions, comforts, prediction of impending disasters, and economic activities such as farming. The collective human efforts towards understanding the ABL are reflected in the enormous amount of literature and computational capacities dedicated to numerical weather modeling. As of today, the National Oceanic and Atmospheric Administration (NOAA) alone hosts a high-performance computing facility capable of operating at 5.78 petaflops, dedicated to run operational weather models of various ranges. Nevertheless, it has nearly been a hundred years since the onset of turbulence research. Lumely and Yaglom \cite{lumely_yaglom_FTC_2001}  while commenting on the current state of knowledge on turbulence have said ``we do have a crude, practical, working understanding of many turbulence phenomena but certainly nothing approaching a comprehensive theory, and nothing that will provide predictions of an accuracy demanded by designers". 


There are many competing definitions of turbulence and some of them reiterated in \cite{hinze_book_75, tsinober_book_2001} emphasize that no definition of turbulence is sufficiently adequate to capture the essence of this complex phenomena. Turbulence can be better understood by its characteristics and properties. The properties as have been described by \citet{tsinober_book_2001} are briefly mentioned here: 
\begin{itemize}
	\item Turbulence possess  intrinsic spatio-temporal randomness and irregularity
	\item Turbulence encompasses extremely wide range of strongly interacting scales
	\item The details of turbulent flows are extremely sensitive to perturbations 
	\item Turbulent flows are dissipative. Sustenance of turbulence requires external energy input
	\item Turbulent flows are always three-dimensional and rotational
\end{itemize}
It is apparent that turbulence dictates mixing processes at multitudes of scales, be it mixing of momentum or scalar such as temperature, particulates. The range of scales that might be present in any flow can be quantified by the non-dimensional Reynold's number $Re$ ($Re = v_c L_c/\nu$ where, $v_c$ is the characteristic velocity, $L_c$ is the characteristic length, and $\nu$ is the kinematic viscosity). In the ABL $Re$ is usually as high as $10^6$. According to the definition of $Re$ this means that the largest scale in ABL can be as large as $10^6$ times the smallest scale. In this study we are interested in the Large and Very Large Scale Motions (LSMs \& VLSMs) that can span hundred of meters to several kilometres in length scale. At these large scales viscosity does not play any significant role. Thus, studying inviscid flow is sufficient to understand the turbulence dynamics of LSMs and VLSMs. However, studying turbulence dynamics at any scale in the ABL necessitates taking into consideration of a wide variety of forcing and boundary/surface conditions. ABL flow is influenced strongly by rotation of the earth and density gradient in the wall-normal direction known as stratification. The surface condition in the ABL can be quite diverse in real life situation. Various types of land cover and surface heating due to solar radiation needs to be accounted for in the study of a real life ABL flow. However, in order to study only the effect of a single forcing or boundary condition all other force and boundary conditions must be kept constant and invariant over the time period considered. Numerical simulation lends itself as a suitable method to study the effect of a particular external force or boundary condition. So, we generated ABL flow field with numerical simulations. In our study we explore the effect of rotation of the earth on turbulent dynamics of the VLSMs and LSMs.    

In this present study we have put forth our efforts to re-examine turbulence characteristics of LSMs and VLSMs. We also attempted to add new insights, and methods to existing tools and techniques to identify VLSMs and LSMs. In the second chapter we introduce a new technique based on common image processing algorithms to identify and characterize VLSMs. In the third chapter we assess the interactions of VLSMs and LSMs with other smaller scales in order to understand the dynamical importance of these large scale motions on overall turbulence dynamics. The fourth chapter delineates application of dynamic mode decomposition to characterize VLSMs. Especially the properties of VLSMs deduced from conditional averaging procedures or other filtering techniques can be compared with DMD results and conclusions are made about organization of VLSMs. 

\bibliographystyle{unsrtnat}
\bibliography{MyThesisRefs}
\clearpage